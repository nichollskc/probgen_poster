%%%%%%%%%%%%%%%%%%%%%%%%%%%%%%%%%%%%%%%%%
% Jacobs Landscape Poster
% LaTeX Template
% Version 1.1 (14/06/14)
%
% Created by:
% Computational Physics and Biophysics Group, Jacobs University
% https://teamwork.jacobs-university.de:8443/confluence/display/CoPandBiG/LaTeX+Poster
%
% Further modified by:
% Nathaniel Johnston (nathaniel@njohnston.ca)
%
% This template has been downloaded from:
% http://www.LaTeXTemplates.com
%
% License:
% CC BY-NC-SA 3.0 (http://creativecommons.org/licenses/by-nc-sa/3.0/)
%
%%%%%%%%%%%%%%%%%%%%%%%%%%%%%%%%%%%%%%%%%

%----------------------------------------------------------------------------------------
%	PACKAGES AND OTHER DOCUMENT CONFIGURATIONS
%----------------------------------------------------------------------------------------

\documentclass[final]{beamer}

\usepackage[scale=1.15]{beamerposter} % Use the beamerposter package for laying out the poster


\usetheme{confposter} % Use the confposter theme supplied with this template

\setbeamercolor{block title}{fg=jblue,bg=white} % Colors of the block titles
\setbeamercolor{block body}{fg=black,bg=white} % Colors of the body of blocks
\setbeamercolor{block alerted title}{fg=white,bg=dblue!70} % Colors of the highlighted block titles
\setbeamercolor{block alerted body}{fg=black,bg=dblue!10} % Colors of the body of highlighted blocks
% Many more colors are available for use in beamerthemeconfposter.sty

%-----------------------------------------------------------
% Define the column widths and overall poster size
% To set effective sepwid, onecolwid and twocolwid values, first choose how many columns you want and how much separation you want between columns
% In this template, the separation width chosen is 0.024 of the paper width and a 4-column layout
% onecolwid should therefore be (1-(# of columns+1)*sepwid)/# of columns e.g. (1-(4+1)*0.024)/4 = 0.22
% Set twocolwid to be (2*onecolwid)+sepwid = 0.464
% Set threecolwid to be (3*onecolwid)+2*sepwid = 0.708

\newlength{\sepwid}
\newlength{\onecolwid}
\newlength{\twocolwid}
\newlength{\threecolwid}
\setlength{\paperwidth}{46.8in} % A0 width: 46.8in
\setlength{\textwidth}{44.8in}
\setlength{\paperheight}{33.1in} % A0 height: 33.1in
\setlength{\textwidth}{31.1in}
\setlength{\sepwid}{0.024\paperwidth} % Separation width (white space) between columns
\setlength{\onecolwid}{0.22\paperwidth} % Width of one column
\setlength{\twocolwid}{0.464\paperwidth} % Width of two columns
\setlength{\threecolwid}{0.708\paperwidth} % Width of three columns
\setlength{\topmargin}{-0.5in} % Reduce the top margin size

%-----------------------------------------------------------


\usepackage{booktabs} % Top and bottom rules for tables

\makeatletter
\let\@@magyar@captionfix\relax
\makeatother

\usepackage{graphicx}
\usepackage{subfig}

% Matrix decomposition diagram
\usepackage{stackengine}
\stackMath
\newlength\matfield
\newlength\tmplength
\def\matscale{1.}
\newcommand\dimbox[3]{%
  \setlength\matfield{\matscale\baselineskip}%
  \setbox0=\hbox{\vphantom{X}\smash{#3}}%
  \setlength{\tmplength}{#1\matfield-\ht0-\dp0}%
  \fboxrule=1pt\fboxsep=-\fboxrule\relax%
  \fbox{\makebox[#2\matfield]{\addstackgap[.5\tmplength]{\box0}}}%
}
\newcommand\raiserows[2]{%
   \setlength\matfield{\matscale\baselineskip}%
   \raisebox{#1\matfield}{#2}%
}
\newcommand\matbox[5]{
  \stackunder{\dimbox{#1}{#2}{$#5$}}{\scriptstyle(#3\times #4)}%
}
\parskip 1em

\usepackage{siunitx} % For scientific notation of p-values using \num

\usepackage{tabularx}
% Tables with merged vertical cells
\usepackage{multirow}

\setbeamerfont{caption}{size=\footnotesize}

% set colors for alerted blocks (blocks with frame)
\setbeamercolor{block alerted title}{fg=white,bg=nred}
\setbeamercolor{block alerted body}{fg=black,bg=nred!10}

%----------------------------------------------------------------------------------------
%	TITLE SECTION
%----------------------------------------------------------------------------------------

\newcommand{\samelineand}{\qquad}

\title{Comparison of sparse biclustering algorithms for gene expression datasets} % Poster title
\author[shortname]{Katherine Nicholls \inst{1} \inst{2} and Chris Wallace \inst{1} \inst{2}}

\institute[shortinst]{\inst{1} Cambridge Institute for Therapeutic Immunology and Infectious Disease, University of Cambridge, Cambridge, CB2 0AW, UK \inst{2} MRC Biostatistics Unit, Cambridge Biomedical Campus, Forvie Site, Robinson Way, Cambridge, CB2 0SR, UK}

%----------------------------------------------------------------------------------------

\begin{document}

\addtobeamertemplate{block end}{}{\vspace*{2ex}} % White space under blocks
\addtobeamertemplate{block alerted end}{}{\vspace*{2ex}} % White space under highlighted (alert) blocks

%\setlength{\belowcaptionskip}{2ex} % White space under figures
\setlength\belowdisplayshortskip{2ex} % White space under equations


\begin{frame}[t] % The whole poster is enclosed in one beamer frame

\begin{columns}[t] % The whole poster consists of three major columns, the second of which is split into two columns twice - the [t] option aligns each column's content to the top

\begin{column}{\sepwid}\end{column} % Empty spacer column

\begin{column}{\onecolwid} % The first column

%----------------------------------------------------------------------------------------
%	INTRODUCTION
%----------------------------------------------------------------------------------------

\begin{block}{Why biclustering?}

\textbf{Groups of genes that covary} in a \textbf{subset of the samples}.

\begin{itemize}
    \item Detects patterns not visible with gene clustering
    \item Provides link between samples and gene groups
    \item Adjusts for confounders
\end{itemize}

\begin{figure}
\begin{minipage}{1\linewidth}
    \centering
\subfloat[Original matrix \label{fig:raw}]{
    \includegraphics[width=0.2\textwidth]{plots/biclustering_diagrams/raw.png}}
\subfloat[Clustering genes\label{fig:rows}]{
    \includegraphics[width=0.2\textwidth]{plots/biclustering_diagrams/rows.png}}
\subfloat[Clustering samples\label{fig:cols}]{
    \includegraphics[width=0.2\textwidth]{plots/biclustering_diagrams/cols.png}}
\end{minipage}
\\
\begin{minipage}{1\linewidth}
    \centering
\subfloat[Biclustering\label{fig:biclusters}]{
    \includegraphics[width=0.85\textwidth]{plots/biclustering_diagrams/biclusters.png}}
\end{minipage}
\caption{Illustration of different kinds of clustering. The same matrix is used in each case, with rows as genes and columns as samples. Only biclustering captures the true structure of the dataset.}
\label{fig:caption}
    \end{figure}

\end{block}

%------------------------------------------------



%----------------------------------------------------------------------------------------
%	Study aims
%----------------------------------------------------------------------------------------


\begin{block}{Novel study features}

\begin{itemize}
    \item \textbf{Algorithm classes} not previously included in comparison studies
    \item \textbf{Range of complexity} of simulated datasets
    \item \textbf{Direct evaluation} of biclustering on \textbf{real datasets}
\end{itemize}

\end{block}

%----------------------------------------------------------------------------------------



%----------------------------------------------------------------------------------------
%	ALGORITHMS
%----------------------------------------------------------------------------------------

\begin{block}{Algorithm classes}

\begin{table}[t!]
    \caption{Overview of the four classes of biclustering algorithm included.}

    \begin{tabular}{ l | l }
\textbf{Class} & \textbf{Advantages} \\ \hline
\textbf{\textit{Popular}} & \multirow{2}{0.6 \textwidth}{Benchmark - in previous comparison studies} \\
     FABIA, Plaid & \\ \hline
    \textbf{\textit{NMF}} & \multirow{2}{0.6 \textwidth}{Fast, interpretable} \\
    nsNMF, SNMF & \\ \hline
    \textbf{\textit{Tensor}} & \multirow{2}{0.6 \textwidth}{Share information across cell types} \\
    MultiCluster, SDA & \\ \hline
    \textbf{\textit{Adaptive}} & \multirow{2}{0.6 \textwidth}{Mixture of sparse and dense factors, learn K automatically} \\
    BicMix, SSLB & \\ \hline
\end{tabular}
\end{table}

\end{block}

\end{column} % End of the first column

\begin{column}{\sepwid}\end{column} % Empty spacer column

\begin{column}{\onecolwid} % The second column

%----------------------------------------------------------------------------------------
%	RESULTS
%----------------------------------------------------------------------------------------

\begin{block}{Results}

\begin{figure}
\includegraphics[width=0.9 \textwidth]{plots/summary_clust_err_best_theoretical_K_init.pdf}
\caption{Caption}
\end{figure}

\begin{figure}
\includegraphics[width=0.9 \textwidth]{plots/threshold_adjusted_redundancy_mean_lines.pdf}
\caption{Caption}
\end{figure}

\begin{figure}
\includegraphics[width=0.9 \textwidth]{plots/compare_samegenes_K_50_datasets_ko_traits_nz_alpha_0-05.pdf}
\caption{Caption}
\end{figure}

\end{block}

\end{column} % End of column 2

\begin{column}{\sepwid}\end{column} % Empty spacer column
\begin{column}{\onecolwid} % The third column

%----------------------------------------------------------------------------------------
%	RESULTS
%----------------------------------------------------------------------------------------

\begin{block}{Results}


\begin{figure}
\includegraphics[width=0.9 \textwidth]{plots/similarity_methods_K.pdf}
\caption{Caption}
\end{figure}

\begin{figure}
\includegraphics[width=0.9 \textwidth]{plots/IMPC_comp_reqs_s_against_K.pdf}
\caption{Caption}
\end{figure}

\begin{figure}
\includegraphics[width=0.9 \textwidth]{plots/pathway_enrichment_num_unique_best_pathways.pdf}
\caption{Caption}
\end{figure}
\end{block}

%----------------------------------------------------------------------------------------

\end{column} % End of column 3

\begin{column}{\sepwid}\end{column} % Empty spacer column
\begin{column}{\onecolwid} % The fourth column

%----------------------------------------------------------------------------------------
%	CONCLUSION
%----------------------------------------------------------------------------------------

\begin{block}{Conclusion}

\begin{itemize}
    \item Post-processing thresholding essential for most algorithms
    \item Adaptive algorithms best on dataset with unknown K and without processing
    \item NMF algorithms have potential - fast, robust and nsNMF performed well
    \item Little benefit found for tensor algorithms
    \item Introduced metrics for real datasets where truth is not known
\end{itemize}

\end{block}

\begin{alertblock}{Preprint}
For more details, see the preprint on bioRxiv: \\
\small \url{https://doi.org/10.1101/2020.12.15.422852}
\end{alertblock}

%----------------------------------------------------------------------------------------
%	REFERENCES
%----------------------------------------------------------------------------------------

\begin{block}{References}

\nocite{*} % Insert publications even if they are not cited in the poster
{\bibliographystyle{plain}
\bibliography{poster}}

\end{block}

%----------------------------------------------------------------------------------------
%	ACKNOWLEDGEMENTS
%----------------------------------------------------------------------------------------

\begin{center}
\begin{tabular}{ccccc}
\includegraphics[height=22mm]{./Cambridge_University_CMYK.eps} & \hspace{15mm} & \includegraphics[height=22mm]{mrc_logo.jpg} & \hspace{15mm} & \includegraphics[height=22mm]{wellcome-logo-black.jpg}
\end{tabular}
\end{center}

%----------------------------------------------------------------------------------------

\end{column} % End of the fourth column

\end{columns} % End of all the columns in the poster

\end{frame} % End of the enclosing frame

\end{document}
